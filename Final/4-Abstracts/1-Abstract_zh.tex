%----------------------------------------------------------------------
% 中文摘要
%----------------------------------------------------------------------

% 把中文摘要寫在裡面
\begin{zhAbstract}

化合物半導體,如氮化鎵、氮化矽和碳化矽等材料,因其寬能隙、高熱導率以及優異的電性與光電特性,在矽光子學、功率元件與汽車電子用品等領域變得越趨重要。然而,這些材料在生長與製程中容易形成各類缺陷,缺陷將影響晶體的對稱性與局部應變場,進而改變材料的光電性質,是限制元件良率與可靠度的重要因素。

本研究以第一性原理(First-principles)計算為基礎,探討化合物半導體中缺陷對二次諧波產生效應(Second Harmonic Generation, SHG)的影響。由於SHG與晶體的對稱性高度相關,當晶體存在堆疊錯誤或差排等缺陷時,其非線性極化張量與SHG訊號的強度將隨之改變。我們以氮化鎵作為研究對象,模擬缺陷結構下的應變分布以及其對SHG效應的影響,並分析缺陷導致的晶體對稱性破壞與非線性光學效應之間的關聯。
本研究結果有助於理解缺陷如何影響材料性質,進而為非破壞性缺陷檢測方法的發展提供基礎。


    % 這個Command會自動幫你把Config裡面設定的東東填進來
    \zhAbsKeywords
\end{zhAbstract}
