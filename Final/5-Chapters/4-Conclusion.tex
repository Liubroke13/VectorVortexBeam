\chapter{結論與未來展望}
\label{ch:conclusion}

% \section{研究結論}
% \subsection{總結缺陷對SHG的影響}
% 晶格中的缺陷,會導致晶體的對稱性改變,而晶體對稱性的改變會導致...。
% \textcolor{red}{因為二次諧波產生效應與晶體的對稱性相關,且在Eq.(2)中的極化率張量也會因結構的對稱性改變,而使其型式改變。但在x,y,z三個方向的單軸應變中,雖然改變其晶體的對稱性,但是在極化率張量的型式上並沒有劇烈的改變。}
% \subsection{討論應力與SHG之間的關聯性}
% 在我的模擬結果中可以看到,當我在不同方向進行拉應力以及壓應力時,二階極化向量的值明顯在拉應力時較壓應力時大。
% \section{未來研究方向}
% \subsection{擴展至其他化合物半導體(如SiC、AlN)}
% 因非線性光學效應是在非中心對稱的晶體結構下會有較明顯的現象,因此我們可以將此方法推廣至不同的具有非中心對稱特性的化合物半導體,如SiC、SiN以及AlN等,這些化合物半導體材料都具有相似的晶體結構,因此缺陷的結構也大同小異,我們就可以利用這樣的特性,使永相同的模擬方法,去預測在不同缺陷的情況下材料的SHG效應的強度,進而去能分析材料中的缺陷類型。
% \subsection{結合實驗驗證數值結果}

% \subsection{探討其他非線性光學效應}

% 4-1
%\section{研究成果總結}
% 根據我們利用單軸應變與雙軸應變的方式對氮化鎵材料進行應變,改變晶體的對稱性,並觀察模擬出的二階極化張量的分布是否與我們事先預測的相同。以及在晶體結構中引入缺陷結構,本研究先引入較容易理解的基面堆疊錯誤(Basal Stacking Faults, BSF),堆疊錯誤是因為在晶體的成長過程中。

% 晶格錯位等缺陷破壞了材料的週期對稱性,因此需要藉由建立超晶胞的建模方法模擬期SHG。在帶有缺陷的晶體中,當時空間的晶格向量變大時,其倒空間的晶格向量會變小,同時布里淵區也會變小。缺陷破壞原晶體的評儀對稱姓,因此需要更大的超晶胞涵蓋缺陷本身,時空間中的計算範圍因此變大,第一原理計算中要處理的原子數變多,增加計算成本。為了降低計算成本,我們將透過研究氮化鎵材料的缺陷結構造成附近原子的應變場分布來計算SHG的訊號變化。因此先利用簡單的單軸應變結構,來驗證對晶胞施加應力、改變其對稱性,是否會改變SHG的訊號強度。

% 首先在x、y、z三個方向進行單軸應變,且不進行鬆弛計算(relaxation),並進行SHG模擬分析,計算經過應變後的氮化鎵晶體的SHG光譜強度與未應變的SHG強度進行比較,並調整入射光的偏振,了解不同方向的線偏振光與應變方向的關係。

% 因為二次諧波產生效應與晶體的對稱性相關,且在式子中的極化率張量也會因結構的對稱性改變而使其形式改變。但在x、y、z三個方向的單軸應變中,雖然改變其晶體的對稱姓,但是在極化率張量的形式上並沒有劇烈的改變。
% 我們可以藉由了解缺陷結構造成的應力場,來預測該種缺陷的二次協波產生光譜強度的分布。

%-------------------------------------------------
% 本研究透過對氮化鎵材料施加單軸應變,改變其對稱性,並觀察模擬所得的二階極化張量(Second-order susceptibility tensor)之分布是否符合預期。同時,我們也在晶體中引入缺陷結構,首先選擇較易理解的基面堆疊錯誤(Basal Stacking Fault, BSF)作為探討對象。堆疊錯誤通常源於在成長晶體的過程中產生的經錯位等缺陷,進而破壞材料原有的週期性對稱。

% 為了模擬含缺陷的二次協波產生效應,我們採用建立超晶胞的方式進行建模。當晶體中出現缺陷,時空間中的晶格向量擴大,對應地倒空間的晶格向量縮小,導致布里淵區變小。由於缺陷會破壞原始晶體的靜巷與旋轉對稱性,因此需要更大的超晶胞來涵蓋缺陷範圍。這使得第一原理計算中須處理的原子數量增加,進一步提升計算成本。

% 為降低計算負擔,我們擬透過研究缺陷所引發的局部應變場,來預測SHG訊號的變化趨勢。為此,我們先針對簡單的單軸應變結構進行測試,驗證對晶胞施加應力並改變對稱姓是否會明顯影響SHG訊號強度。

% 具體而言,我們對晶體在x、y、z三個方向分別施加單軸應變,並在不進行幾何鬆弛條件下,進行SHG模擬分析。我們比較應變後與為應變氮化鎵晶體的SHG光譜強度,並調整入射光的偏振方向,以探討線偏振光方向與應變方向之間的關聯性。

% 由於SHG效應與晶體的對稱姓高度相關,極化率張量的形式亦會隨結構對稱性的改變而發生變化。然而,在本研究師機的x、y、z單軸應變情況下,儘管對晶體對稱姓造成一定程度的影響,極化率張量的非零元排列仍未出現明顯改變。

% 最終,我們希望透過對缺陷結構所引發的應力場進行分析,建立缺陷種類與其SHG光譜強度分布之間的對應關係,進而提供一種無損方式來探測晶體缺陷。
%----------------------------------------------

\section{結論}
% 我們針對化合物半導體氮化鎵在含有缺陷結構的情況下,其SHG效應的變化。利用第一原理軟體進行SHG效應的模擬,並進行SHG強度的偏振分析,了解不同的缺陷結構在不同的入射光以及收光偏振下強度分布的變化。

% 無缺陷的GaN晶體的點群對稱行為$C_{6v}$,根據Neumann principle推出對應的二階極化率張量中的非零值分布,在入射光為垂直入射的情況下,二階極化向量將只有$P_z$分量,而收光也在垂直方向收光,將不會有SHG訊號的強度;而在x-z平面上的斜向入社與收光,則不管是入射P或是S偏振的光,將都會接收到SHG訊號的強度。

% GaN的堆疊錯誤缺陷結構會使得材料的點群對稱性變為$C_{3v}$,會使得原本為零的二階極化率張量元變為非零,使得其SHG響應會與無缺陷不同。當入射光與收光為垂直樣品方向$z$方向時,其SHG強度與方位角$\phi$的變化呈現六重花瓣形。



% 本研究針對化合物半導體材料GaN在不同應變及缺陷結構下的二次諧波產生(SHG)強度進行了系統性分析。藉由理論推導與公式計算,我們深入探討了入射光偏振態、入射角度$\theta$及方位角$\phi$對SHG響應的影響。

% 對於無缺陷的GaN晶體,其對稱性屬於$C_{6v}$點群,二階極化率張量中非零元的特定分布限制了SHG的產生。在垂直入射條件下($\theta=0^\circ$),無論入射光或出射光的偏振態為何,SHG強度均趨近於零,顯示完美晶體結構中雖然其非中心對稱性晶體,但是因其主要產生的極化像量皆在$z$方向,因此若只在$z$方向觀測出射光,便不會有SHG效應的產生。

% 當GaN晶體受到單軸應變導致對稱性改變為$C_{2v}$點群時,極化率張量元雖保有類似非零結構,但因各元素不再相同,導致SHG強度對入射角度及偏振組合呈現微妙變化。然而,垂直入射下SHG信號依舊微弱。

% 相較之下,堆疊錯誤產生的$C_{3v}$點群結構顯著打破了原本的對稱限制,產生不同的非線性光學響應。在垂直入射條件下,SHG強度呈現明顯的方位角$\phi$週期性變化,並且不同偏振組合的強度分布對應於$\sin^2(3\phi)$與$\cos^2(3\phi)$,此特性不僅為堆疊錯誤的敏感指標,也為缺陷結構的光學鑑定提供了有效工具。

% 此外,斜向入射($\phi=0^\circ$)時,SHG強度隨入射角$\theta$變化而展現不同的響應曲線,表示不同點群對應極化率張量元的角度選擇性與耦合機制。該角度依賴性進一步印證了非線性光學過程對晶體結構與入射條件的高度敏感性。

% 綜合上述,本研究所建立的 SHG 強度分析模型,從第一原理出發,結合晶體對稱性與光學偏振態的影響,不僅加深對GaN及類似化合物半導體材料非線性光學特性的理解,也可為實驗中應用 SHG 技術辨識晶體缺陷與結構變化奠定理論基礎。
% 除了本研究聚焦的刃型差排應力場與堆疊錯誤缺陷外,未來亦可進一步延伸至其他缺陷類型的探討,例如實際差排核心結構、摻雜引入的點缺陷等對 SHG 響應的影響,藉此建立更全面的缺陷與非線性光學特性之間的對應關係,進一步提升 SHG 在缺陷檢測上的準確性與應用價值。此外,本研究的方法也能用在其他化合物半導體材料上,未來有機會成為發展非破壞性光學檢測技術的理論基礎。

% %%%%%%%%%%%%%%%%%%%%%%%%%%%%%%%%%%%%%%%%%%%%%%%%%%%%%%%%%%%%%%%%%%%
% 本研究針對化合物半導體氮化鎵(GaN)在不同應變與缺陷結構下的二次諧波產生(SHG)效應進行系統性探討。我們利用第一原理模擬方法,結合理論推導與公式計算,分析了入射光偏振態、入射角度 $\theta$ 與方位角 $\phi$ 對 SHG 強度的影響,並進一步比較了完美晶體與缺陷結構在 SHG 響應上的差異。

% 對於無缺陷的 GaN 晶體,其對稱性屬於 $C_{6v}$ 點群。由 Neumann principle 可得其二階極化率張量的非零分布,顯示在垂直入射條件下($\theta=0^\circ$),二階極化向量僅存在 $P_z$ 分量,因此若僅在 $z$ 方向收光,SHG 強度將趨近於零;然而,在 $x$–$z$ 平面上進行斜向入射與收光時,無論入射光為 P 偏振或 S 偏振,皆能觀測到 SHG 訊號。當晶體受到單軸應變導致對稱性降低至 $C_{2v}$ 點群時,雖然張量元仍維持類似結構,但因元素間不再完全相等,SHG 強度對入射角度及偏振組合呈現細微變化,然而垂直入射下的信號依然微弱。

% 相比之下,堆疊錯誤缺陷使晶體對稱性轉變為 $C_{3v}$,顯著打破了原本的對稱限制,導致原本為零的張量元成為非零值,並產生不同的非線性光學響應。在垂直入射條件下,SHG 強度隨方位角 $\phi$ 呈現明顯的週期性變化,對應於 $\sin^2(3\phi)$ 與 $\cos^2(3\phi)$ 的六重花瓣分布,此特徵不僅能作為堆疊錯誤的敏感指標,也為缺陷結構的光學鑑定提供了有效依據。此外,在斜向入射($\phi=0^\circ$)時,SHG 強度隨入射角 $\theta$ 的變化展現出不同的響應曲線,進一步凸顯了非線性光學過程對晶體結構與入射條件的高度敏感性。

% 綜合上述結果,本研究所建立的 SHG 強度分析模型,從第一原理出發,結合晶體對稱性與光學偏振條件的影響,不僅深化了對 GaN 及相關化合物半導體非線性光學特性的理解,也為實驗中利用 SHG 技術進行缺陷鑑定提供了理論依據。除了聚焦於刃型差排應力場與堆疊錯誤缺陷外,未來研究亦可延伸至其他缺陷型態,例如差排核心結構或摻雜引入的點缺陷,藉以建立更完整的缺陷–非線性光學特性對應關係,進一步提升 SHG 在缺陷檢測上的準確性與應用價值。同時,本研究的方法也可推廣至其他化合物半導體材料,有望成為發展非破壞性光學檢測技術的重要理論基礎。
% %%%%%%%%%%%%%%%%%%%%%%%%%%%%%%%%%%%%%%%%%%%%%%%%%%%%%%%%%%%%%%%%%%%

% 本研究是針對化合物半導體GaN在不同缺陷結構下,對非線性光學效應SHG的影響。我們利用第一原理模擬方法,結合理論推倒以及公式計算,分析了入射光偏振態、入射角度$\theta$與方位角$\phi$對SHG強度的影響,並進一步比較完美晶體與缺陷結構在SHG響應上的差異。

% 完美的GaN晶體的對稱性$C_{6v}$所對應的二階極化率張量的形式,會使得二階極化向量在垂直入射時只存在$P_z$方向的分量,因此若在$z$方向收光,SHG強度江維玲;然而在$x-z$平面上進行斜向入射與收光時,無論入射光為P偏振或S偏振,皆能觀測到SHG訊號。當晶體受到單軸與雙軸應變導致對稱性變為$C_{2v}$點群時,雖然張量仍維持類似結構,但張量元間不再完全相等,SHG強度對入射角度及偏振組合會呈現些微的改變,但是在垂直入射下,SHG信號依然為零。

% 相較之下,$I_1$與$I_2$型的堆疊錯誤缺陷會使勁體的對稱性變為$C_{3v}$,會使得原本為零的張量元成為非零值,並產生不同的非線性光學響應。在垂直入射條件下,SHG強度會隨方位角$\phi$呈現明顯的周期性變化,$\sin^2(3\phi)$ 與 $\cos^2(3\phi)$ 的六重花瓣分;在斜向入射時,SHG強度隨入射角$\theta$的變化會因為極化向量含有$P_x$、$P_y$分量,而SHG強度分布與完美晶體的分不同。

% 綜合上述結果,我們建立了以第一原理出發結合晶體對稱性以及光學偏振條件的影響,可以讓我們理解GaN及類似化合物半導體材料的缺陷以及非線性光學的特性,。
% % 在第三章中,我們分別將不同缺陷結構的氮化鎵結構進行SHG的模擬,並進行SHG強度的偏振分析。
% %%%%%%%%%%%%%%%%%%%%%%%%%%%%%%%%%%%%%%%%%%%%%

% 本研究利用第一原理模擬方法,探討了氮化鎵(GaN)在不同缺陷結構下的二次諧波產生(Second Harmonic Generation, SHG)效應,並分析了入射光偏振態、入射角度 $\theta$ 與方位角 $\phi$ 對 SHG 響應的影響。藉由比較完美晶體、堆疊錯誤缺陷與edge dislocation所造成的應變結構,
% 本研究說明了晶體對稱性與缺陷對非線性光學響應的影響。

% 在完美的 $C_{6v}$ 對稱 GaN 晶體中,SHG 僅能在斜向入射的條件下被觀測到,而垂直入射時則幾乎完全消失。這顯示在高對稱性的晶體中,SHG 響應受到嚴格限制,其非線性光學特性高度依賴於入射角與偏振條件。

% 相較之下,堆疊錯誤缺陷($I_1$ 與 $I_2$型)使晶體對稱性由 $C_{6v}$ 降為 $C_{3v}$,不僅使原本為零的張量元成為非零,還在垂直入射條件下引入顯著的方位角$\phi$依賴性。SHG 強度會呈現六重對稱分布,反映出缺陷結構對非線性響應的敏感影響。這說明堆疊錯誤是能夠有效改變並放大 SHG 響應的重要缺陷類型,亦提供了一個可藉由偏振分布特徵來識別的實用途徑。

% 另一方面,edge dislocation所引起的應變結構會將對稱性降為 $C_{2v}$,使張量元之間的關係被打破,進而影響 SHG 響應。然而,此影響主要表現在入射角與偏振組合的細微差異上,整體強度改變有限,且在垂直入射下依然無法產生明顯信號。這意味著,應變效應雖然能改變 SHG 行為,僅會影響SHG強度大小,但其敏感度與影響程度均不及堆疊錯誤。

% 綜合而言,本研究建立了從晶體對稱性出發,結合偏振條件與第一原理模擬的分析框架,清楚揭示了不同缺陷類型對 GaN晶體 SHG 特性的影響。這些結果不僅深化了對化合物半導體非線性光學行為的理解,也為發展非破壞性缺陷檢測方法提供了理論依據。未來,這樣的研究可進一步延伸至其他化化物半導體材料與其他異質結構(如 SiC、SiN 或多層 GaN/AlGaN 系統),以探討在複雜材料環境下缺陷對 SHG 的影響。同時,結合實驗量測與模擬的交叉驗證,將有助於建立更精準的缺陷探測技術。此外,若能將 SHG 對缺陷敏的感度應用於半導體製程品質控管與新型光電元件設計,將為非線性光學與先進半導體技術之間建立更緊密的連結。

%%%%%%%%%%%%%%%%%%%%%%%%%%%%%%%%%%%%%%%%%

本研究利用第一原理模擬方法,探討了氮化鎵(GaN)在不同缺陷結構下的二次諧波產生(Second Harmonic Generation, SHG)效應,並分析了入射光偏振態、入射角度 $\theta$ 與方位角 $\phi$ 對 SHG 響應的影響。藉由比較完美晶體、堆疊錯誤缺陷與刃狀差排所造成的應變結構,本研究說明了晶體對稱性與缺陷對非線性光學響應的影響。

在完美的 $C_{6v}$ 對稱 GaN 晶體中,SHG 僅能在斜向入射條件下被觀測到,而垂直入射時則幾乎完全消失。這顯示在高對稱性的晶體中,SHG 響應受到嚴格限制,其非線性光學特性高度依賴於入射角與偏振條件。

相較之下,堆疊錯誤缺陷($I_1$ 與 $I_3$ 型)使晶體對稱性由 $C_{6v}$ 降為 $C_{3v}$,不僅使原本為零的張量元成為非零,還在垂直入射條件下引入顯著的方位角 $\phi$ 依賴性。SHG 強度呈現六重對稱分布,反映出缺陷結構對非線性響應的敏感影響。這說明堆疊錯誤是能夠有效改變並放大 SHG 響應的重要缺陷類型,亦提供了一個可藉由偏振分布特徵來識別的實用途徑。

另一方面,刃狀差排所引起的應變結構會將對稱性降為 $C_{2v}$,使張量元之間的關係被打破,進而影響 SHG 響應。然而,此影響主要表現在入射角與偏振組合的細微差異上,整體強度改變有限,且在垂直入射下依然無法產生明顯信號。應變雖然能改變 SHG 行為,但影響程度有限,其靈敏度不及堆疊錯誤對SHG強度造成的影響。

綜合而言,本研究建立了從晶體對稱性出發,結合偏振條件與第一原理模擬的分析,說明了不同缺陷類型對 GaN 晶體 SHG 特性的影響。這些結果不僅深化了對化合物半導體非線性光學行為的理解,也為發展非破壞性的缺陷檢測方法提供了理論依據。未來,這樣的研究可進一步延伸至其他化合物半導體材料(如 SiC、SiN等材料),及不同類型的缺陷結構,以探討不同缺陷及材料對 SHG 的影響。同時,結合實驗量測與模擬的比對,將有助於建立更精準的缺陷探測技術。此外,若能將 SHG 對缺陷的靈敏度應用於半導體製程品質控管與新型光電元件設計,將有助於在非線性光學與先進半導體技術之間建立更緊密的連結。

%%%%%%%%%%%%%%%%%%%%%%%%%%%%%%%%%%%%%%%%%


% % 4-2
% \section{未來展望}
% 在本研究中,我們初步探討了化合物半導體氮化鎵材料在應變與缺陷情況下的二次諧波產生效應,並驗證其與晶體對稱性變化之間的關聯。未來將近一部擴展模擬範圍,探討更多類型的結構缺陷,如穿通差排(Threading Dislocation)等,並分析其所引發的局部應變場與非線性光學響應之間的對應關係。此外。最終,我們期望本研究所建立的方法能應用於發展非破壞性的光學檢測技術,為化合物半導體材料的品質評估與晶體結構診斷提供有效的理論依據與技術支持。

% \section{未來展望}

% 本研究聚焦於GaN材料不同對稱性與缺陷結構對二次諧波產生(SHG)強度的影響,展現了SHG技術在晶體結構分析中的潛力。未來可從以下幾個方向進一步拓展與深化:

% 首先,除目前探討的刃型差排的應力場以及堆疊錯誤缺陷外,可進一步研究其他缺陷類型如實際的差排大小、摻雜與複合缺陷對SHG響應的影響,從而建立更全面的缺陷-光學關係模型,提升缺陷檢測的靈敏度與準確度。
% % 其次,可結合溫度、應力及電場等外部調控參數,探討這些因素對GaN非線性光學特性的調控機制,期望開發出可控的光學功能材料,促進高性能光電元件的設計與應用。
% 也可將本研究理論模型與實驗數據進行更緊密的結合,通過定量比較與反向推演,優化模型參數與計算方法,提升模擬預測的準確性與實用價值。
% 最後,鑑於SHG技術具備非破壞、靈敏度高等優點,未來可拓展至其他寬禁帶半導體及複合材料系統,推動先進半導體材料的結構與品質分析,為光電子產業帶來革新性助力。
% 綜合而言,本研究成果為GaN材料及類似系統的非線性光學特性提供了理論支撐,未來透過多方整合與深入探索,勢必能加速相關材料與器件的發展與應用。





%In this section, 整理效能評估.

%下面是subfigure的範例 (其實我個人不常用這個語法, 我都直接在繪圖工具上把圖片整合在一起 XD)

%\begin{figure}
%     \centering
%     \begin{subfigure}[]{0.3\textwidth}
%         \centering
%         \includegraphics[width=\textwidth]{fig1.png}
%         \caption{figure 1}
%         \label{fig:11111}
%     \end{subfigure}
%     \hfill
%     \begin{subfigure}[]{0.3\textwidth}
%         \centering
%         \includegraphics[width=\textwidth]{fig2.png}
%         \caption{figure 2}
%         \label{fig:22222}
%     \end{subfigure}
%     \hfill
%     \begin{subfigure}[]{0.3\textwidth}
%         \centering
%         \includegraphics[width=\textwidth]{fig3.png}
%         \caption{figure 3}
%         \label{fig:33333}
%     \end{subfigure}
%        \caption{Three simple graphs}
%        \label{fig:three graphs}
%\end{figure}